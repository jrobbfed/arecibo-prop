\documentclass[preprint]{/Users/jesse/tex/aastex_JRF}

\usepackage{graphicx}
\usepackage[space]{grffile}
\usepackage{latexsym}
\usepackage{amsfonts,amsmath,amssymb}
\usepackage{url}
\usepackage[utf8]{inputenc}
\usepackage{fancyref}
\usepackage{hyperref}
\hypersetup{colorlinks=false,pdfborder={0 0 0},}
\usepackage{textcomp}
\usepackage{longtable}
\usepackage{multirow,booktabs}



\begin{document}

\title{Deep Pulsar Searches in Ultra-Faint Dwarf Galaxies}
\author{Jesse Feddersen\\ \emph{Department of Astronomy, Yale University}\\~\\}

\maketitle 


Our search for pulsars in ultra-faint dwarf galaxies will address three main science goals:
\begin{enumerate}
\item The discovery or a pulsar in a UFD would be the \textbf{first known extragalactic pulsar outside of the Magellanic Clouds}.
\item Placing observational limits on the pulsar population in UFDs will provide the first \textbf{constraint on the high-mass initial mass function} (IMF) of the oldest dynamically unevolved stellar populations.
\item By measuring the dispersion of the pulses from a pulsar in a UFD, we \textbf{probe the electron density of the intergalactic medium} towards that line of sight.
\end{enumerate}
We justify each of these goals in turn.

\section{First of its Kind}
%Put a short history of pulsar surveys here. March to farther distances. 
%Summarize the pulsar population, disk, GCs, Magellanic Clouds, link to expected properties of UFD pulsar (MSP)


\section{Constraining the High-Mass IMF}
%Short history of IMF, recent evidence of variation, Marla's work on low-mass IMF (bottom-light) in UFDs. Distribution of neutron stars from different IMF forms.
The initial mass function is the distribution of stellar masses in a stellar population at the beginning of star formation. The IMF determines the evolution of the population and is a crucial input in models of synthetic stellar populations. The form of the IMF affects many galaxy parameters derived from stellar population synthesis. The form of the IMF also places a constraint on star formation theory, which must predict the observed IMF. In the Milky Way, the IMF is typically parametrized by the similar~\citet{Kroupa01} or~\citet{Chabrier03} laws with little variation across a range of star-forming environments~\citep{Bastian10}. A departure from this ``universal'' IMF indicates a star formation process that depends on environment.

\textbf{Cite evidence of low-mass IMF variation in UFDs here.} All work on the IMF in UFDs to date has focused on the extant stars in the galaxies with $M < 0.8M_{\odot}$. All higher mass stars are now stellar remnants, and we aim to place an observational constraint on the number of stellar remnants through pulsar searches.

Pulsars are rapidly rotating neutron stars. We can predict the number of neutron stars in a UFD by adopting the simple prescription that stars with masses between $9 < M/M_{\odot} < 25$ become neutron stars after undergoing core-collapse supernovae~\citep{Heger03}. In the limit of an infinite stellar population, the IMF uniquely determines the fraction of stars that become neutron stars. For the small masses of UFDs, the effects of stochastic IMF sampling can have a significant impact on the actual distribution of stellar masses~\citep{Hernandez12}. In Figure~\ref{fig:ns_frac}, I show the distribution of the fraction of stars that become neutron stars for 200 realizations of a $1000~M_{\odot}$ galaxy, assuming either Kroupa or~\citet{Salpeter55} IMF.

\begin{figure}[h!]
\begin{center}
\includegraphics[width=0.98\columnwidth]{figures/ns_frac.pdf}
\caption{Replace this text with your caption}
\end{center}
\end{figure}



\section{Probing the Intergalactic Medium}
%dispersion measure. 

\bibliographystyle{/Users/jesse/tex/astronat/apj/apj.bst} 
\bibliography{/Users/jesse/tex/astronat/apj/apj-jour.bib,/Users/jesse/Dropbox/all.bib}

\end{document}

